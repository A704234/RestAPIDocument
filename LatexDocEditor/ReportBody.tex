\documentclass{report}
\usepackage{dirtree}
\usepackage{listings}
\usepackage{color}
\usepackage{graphicx}
% Subtract 1 from counters that are used
\renewcommand{\thesection}{\thechapter.\number\numexpr\value{section}-1\relax}
\renewcommand{\thesubsection}{\thesection.\number\numexpr\value{subsection}-1\relax}
\renewcommand{\thesubsubsection}{\thesubsection.\number\numexpr\value{subsubsection}-1\relax}


\definecolor{dkgreen}{rgb}{0,0.6,0}
\definecolor{gray}{rgb}{0.5,0.5,0.5}
\definecolor{mauve}{rgb}{0.58,0,0.82}

\lstset{frame=tb,
	language=java,
	aboveskip=3mm,
	belowskip=3mm,
	showstringspaces=false,
	columns=flexible,
	basicstyle={\small\ttfamily},
	numbers=none,
	numberstyle=\tiny\color{gray},
	keywordstyle=\color{blue},
	commentstyle=\color{dkgreen},
	stringstyle=\color{mauve},
	breaklines=true,
	breakatwhitespace=true,
	tabsize=3
}

\setcounter{secnumdepth}{3}

\setcounter{chapter}{1}% Not using chapters, but they're used in the counters
\begin{document}
			\begin{center}
		{\Large \textbf{\textcolor{blue}{RE}presentational \textcolor{blue}{S}tate \textcolor{blue}{T}ransfer \\\vspace{0.3cm} Web Service}}\\\vspace{1cm}
	\end{center}
	\begin{center}
		\begin{figure}
			\includegraphics[width=\linewidth]{images/FrontPageTitle.png}
		\end{figure}
	\end{center}
	\tableofcontents
	\newpage
	\section{Project Folder Structure}
	\dirtree{%
		.1 HELP.md.
		.1 mvnw.
		.1 mvnw.cmd.
		.1 pom.xml.
		.1 src.
		.2 main.
		.3 java.
		.4 com.
		.5 atos.
		.6 equens.
		.7 worldline.
		.8 SimpleSpringRestAPI.
		.9 controller.
		.10 EmployeeController.java.
		.9 DAO.
		.10 EmployeeDAO.java.
		.9 model.
		.10 Employee.java.
		.9 service.
		.10 EmployeeService.java. 
		.8 SimpleSpringRestApiApplication.java.
		.2 resource.
		.3 application.properties.
		.1 test.
		.2 java.
		.3 com.
		.4 atos.
		.5 equens.
		.6 worldline.
		.7 SimpleSpringRestAPI.
		.8 SimpleSpringRestApiApplicationTests.java.   
	}
	\newpage
	\section{Introduction}
	\newpage
	\section{Maven}
	\newpage
	\section{Rest vs SOAP}
	\newpage
	\section{Project Object Model}	
	
	\begin{lstlisting}[language = xml]
	\\pom.xml
		<?xml version="1.0" encoding="UTF-8"?>
		<project xmlns="http://maven.apache.org/POM/4.0.0" xmlns:xsi="http://www.w3.org/2001/XMLSchema-instance"
			xsi:schemaLocation="http://maven.apache.org/POM/4.0.0 http://maven.apache.org/xsd/maven-4.0.0.xsd">
			<modelVersion>4.0.0</modelVersion>
			<parent>
				<groupId>org.springframework.boot</groupId>
				<artifactId>spring-boot-starter-parent</artifactId>
				<version>2.1.6.RELEASE</version>
				<relativePath/> <!-- lookup parent from repository -->
			</parent>
			<groupId>com.atos.equens.worldline</groupId>
			<artifactId>Messenger</artifactId>
			<version>0.0.1-SNAPSHOT</version>
			<name>MessengerRestSpring</name>
			<description>Messenger Spring Boot Rest  API project for Spring Boot</description>
			<dependencies>
				<dependency>
					<groupId>org.springframework.boot</groupId>
					<artifactId>spring-boot-starter-jersey</artifactId>
				</dependency>
				<dependency>
					<groupId>org.springframework.boot</groupId>
					<artifactId>spring-boot-starter-data-jpa</artifactId>
				</dependency>
				<dependency>
					<groupId>com.oracle</groupId>
					<artifactId>ojdbc14</artifactId>
					<version>10.2.0.4.0</version>
				</dependency>
				<dependency>
					<groupId>org.springframework.boot</groupId>
					<artifactId>spring-boot-starter-test</artifactId>
					<scope>test</scope>
				</dependency>
			</dependencies>
		</project>
	\end{lstlisting}
	The above pom.xml describes a spring boot maven application with minimum dependencies with which we can write the simple spring boot appliaction which illustrates the usange of Rest Web Services and its method.
	Let’s us study the pom.xml file in brief just get to the purpose of each dependencies why being used.
	\subsection{Parent}
	Parent tag in above pom.xml differentiates the spring boot application from a simple maven appliaction. It tells maven that it is Spring boot appliaction asual as asks for transitive dependencies to be avalble.
		\begin{lstlisting}[language=xml]
		<parent>
			<groupId>org.springframework.boot</groupId>
			<artifactId>spring-boot-starter-parent</artifactId>
			<version>2.1.6.RELEASE</version>
			<relativePath/> <!-- lookup parent from repository -->
		</parent>
		\end{lstlisting}
	If you are running spring-boot application for the first time, required dependencies which are mentioned in pom.xml will be downloaded from the maven repository (.m2/repository). Downloaded dependencies will be stored in local repository which will be easy to use  next time withought having to download every time you create new application.
	\subsection{Dependency}
	Dependency tag describes the required maven dependencies. In above pom.xml,I tried to use minimum dependencies which required to illustrate the standard Rest Api using Spring Boot. Let us go through each dependencies and purpose of being used. 
	\subsubsection{Jersey Library}
		\begin{lstlisting}[language=xml]
		<dependency>
			<groupId>org.springframework.boot</groupId>
			<artifactId>spring-boot-starter-jersey</artifactId>
		</dependency>
		\end{lstlisting}
		Jersey is an open source framework for developing RESTful Web Services in Java. It is 		a reference implementation of the Java API for RESTful Web Services (JAX-RS) specification. 
	\subsubsection{Hibernate/JPA }
		\begin{lstlisting}[language=xml]
		
		<dependency>
			<groupId>org.springframework.boot</groupId>
			<artifactId>spring-boot-starter-data-jpa</artifactId>
		</dependency>
		\end{lstlisting}
		The Java Persistence API (JPA) is a Java application programming interface 	specification that describes the management of relational data in applications using Java 	Platform, Standard Edition and Java Platform, Enterprise Edition. 	Persistence in this context 	covers three areas: 
		\begin{itemize}
			\item the API itself, defined in the javax.persistence package
			\item the API itself, defined in the javax.persistence package
			\item object/relational metadata
		\end{itemize}		
	\subsubsection{JDBC Driver}
		\begin{lstlisting}[language=xml]
		<dependency>
			<groupId>com.oracle</groupId>
			<artifactId>ojdbc14</artifactId>
			<version>10.2.0.4.0</version>
		</dependency>
		\end{lstlisting}
	\subsubsection{Junit/Spring Test}
		\begin{lstlisting}[language=xml]
		<dependency>
			<groupId>org.springframework.boot</groupId>
			<artifactId>spring-boot-starter-test</artifactId>
			<scope>test</scope>
		</dependency>
		\end{lstlisting}
		Spring Boot provides a number of utilities and annotations to help when testing your 	application. Test support is provided by two modules: spring-boot-test contains core 	items, and spring-boot-test-autoconfigure supports auto-configuration for tests.
		Most developers use the spring-boot-starter-test “Starter”, which imports both 	Spring Boot test modules as well as JUnit, AssertJ, Hamcrest, and a number of other useful 	libraries.	
	\newpage
	\section{application.properties}
	Properties files are used to keep ‘N’ number of properties in a single file to run the application in a different environment. In Spring Boot, properties are kept in the application.properties file under the classpath.
	\par The application.properties file is located in the src/main/resources directory. The code for this project’s application.properties file is given below.
	\begin{lstlisting}
	# ===============================
	# = DATA SOURCE
	# ===============================
	# Set here configurations for the database connection
	spring.datasource.url=jdbc:mysql://localhost:3306/Test
	spring.datasource.username=root
	spring.datasource.password=root
	spring.datasource.driver-class-name=com.mysql.cj.jdbc.Driver
	# Keep the connection alive if idle for a long time (needed in production)
	spring.datasource.testWhileIdle=true
	spring.datasource.validationQuery=SELECT 1
	# ===============================
	# = JPA / HIBERNATE
	# ===============================
	# Show or not log for each sql query
	spring.jpa.show-sql=true
	# Hibernate ddl auto (create, create-drop, update): with "create-drop" the database
	# schema will be automatically created afresh for every start of application
	spring.jpa.hibernate.ddl-auto=create-drop
	# Naming strategy
	spring.jpa.hibernate.naming.implicit-strategy=org.hibernate.boot.model.naming.ImplicitNamingStrategyLegacyHbmImplspring.jpa.hibernate.naming.physical-strategy=org.springframework.boot.orm.jpa.hibernate.SpringPhysicalNamingStrategy
	# Allows Hibernate to generate SQL optimized for a particular DBMS
	spring.jpa.properties.hibernate.dialect=org.hibernate.dialect.MySQL5InnoDBDialect
	server.port=8089
	\end{lstlisting}
	\section{Annotations}
	\subsection{JPA Annotations}
	\subsubsection{@Entity}
	\subsubsection{@Table}
	\subsubsection{@Column}
	\subsubsection{@Id}
	\subsubsection{@GeneratedValue}
	\subsection{Spring Framework Annotations}
	\subsubsection{@Repository}
	\subsubsection{@Autowired}
	\subsubsection{@Service}
	\subsubsection{@RequestController}
	\subsubsection{@RequestMapping}
	\subsubsection{@RequestMethod}
	\subsubsection{@PathVariable}
	\section{Java Code}
	\subsection{Problem Statement}
	\subsection{Model}
	\begin{lstlisting}
	\\Employee.java
	package com.atos.equens.worldline.SimpleSpringRestAPI.model;
	
	import javax.persistence.Column;
	import javax.persistence.Entity;
	import javax.persistence.GeneratedValue;
	import javax.persistence.GenerationType;
	import javax.persistence.Id;
	import javax.persistence.Table;
	
	@Entity
	@Table(name = "Employee")
	public class Employee {
	
		@Column(name = "ID")
		@Id
		@GeneratedValue(strategy = GenerationType.AUTO)
		private int id;
		@Column(name = "FirstName")
		private String firstName;
		@Column(name = "LastName")
		private String lastName;
		public int getId() {
			return id;
		}
		public void setId(int id) {
			this.id = id;
		}
		public String getFirstName() {
			return firstName;
		}
		public void setFirstName(String firstName) {
			this.firstName = firstName;
		}
		public String getLastName() {
			return lastName;
		}
		public void setLastName(String lastName) {
			this.lastName = lastName;
		}		
	}
	\end{lstlisting}
	\subsection{DAO}
	\begin{lstlisting}
	\\EmployeeDAO.java
	package com.atos.equens.worldline.SimpleSpringRestAPI.DAO;
		
	import org.springframework.data.jpa.repository.JpaRepository;
	import org.springframework.stereotype.Repository;
	import com.atos.equens.worldline.SimpleSpringRestAPI.model.Employee;
		
	@Repository
	public interface EmployeeDAO extends JpaRepository<Employee, Integer>{

	}		
	\end{lstlisting}
	\subsection{Servcie}
	\begin{lstlisting}
	\\EmployeeService.java
	package com.atos.equens.worldline.SimpleSpringRestAPI.service;
	
	import java.util.List;
	import java.util.Optional;
	
	import org.springframework.beans.factory.annotation.Autowired;
	import org.springframework.stereotype.Service;
	
	import com.atos.equens.worldline.SimpleSpringRestAPI.DAO.EmployeeDAO;
	import com.atos.equens.worldline.SimpleSpringRestAPI.model.Employee;
	
	@Service
	public class EmployeeService {
	
		@Autowired
		EmployeeDAO employeeDAO;
	
		public List<Employee> getAllEmployee(){
			return employeeDAO.findAll();
		}
		
		public Employee addEmployee(Employee employee) {
			return employeeDAO.save(employee);
		}
		
		public Optional<Employee> getEmployeeByID(int id) {
			return employeeDAO.findById(id);
		}
		
		public Employee updateEmployee(Employee employee) {
			return employeeDAO.save(employee);
		}
		
		public void deleteEmployeeById(int id) {
			employeeDAO.deleteById(id);
		}
		
		public void deleteAllEmployee() {
			employeeDAO.deleteAll();
		}
	}
	\end{lstlisting}
	\subsection{Controller} 
	\begin{lstlisting}
	\\EmployeeController.java
	package com.atos.equens.worldline.SimpleSpringRestAPI.controller;
		
	import java.util.List;
		
	import org.springframework.beans.factory.annotation.Autowired;
	import org.springframework.web.bind.annotation.RequestMapping;
	import org.springframework.web.bind.annotation.RequestMethod;
	import org.springframework.web.bind.annotation.RestController;
		
	import com.atos.equens.worldline.SimpleSpringRestAPI.model.Employee;
	import com.atos.equens.worldline.SimpleSpringRestAPI.service.EmployeeService;
		
	@RestController
	@RequestMapping("/employee")
	public class EmployeeController {
	
		@Autowired
		EmployeeService employeeService;
	
		@RequestMapping(value = "/getallemployees", method=RequestMethod.GET)
		public List<Employee> getAllEmployee(){
			return employeeService.getAllEmployee();
		}
		
			
		@RequestMapping(
			value = "/addemployee", 
			method = RequestMethod.POST, 
			consumes = MediaType.APPLICATION_JSON_VALUE, 
			produces = MediaType.APPLICATION_JSON_VALUE
		)
		public Employee addEmployee(@RequestBody Employee employee){
			return employeeService.addEmployee(employee);
		}
		
		@RequestMapping(
			value = "/updateemployee", 
			method = RequestMethod.PUT,
			consumes = MediaType.APPLICATION_JSON_VALUE, 
			produces = MediaType.APPLICATION_JSON_VALUE
		)
		public Employee updateEmployee(@RequestBody Employee employee) {
			return employeeService.updateEmployee(employee);
		}
		
		@RequestMapping(value = "/{id}", method = RequestMethod.GET)
			public Optional<Employee> getEmployeeById(@PathVariable int  id){
				return employeeService.getEmployeeByID(id);
		}
		
		@RequestMapping(value = "/deleteallemployee", method = RequestMethod.DELETE)
		public void deleteAllEmployee(){
			employeeService.deleteAllEmployee();
		}
		
		@RequestMapping(value = "/{id}", method = RequestMethod.DELETE)
		public void deleteEmployeeById(@PathVariable int  id){
			employeeService.deleteEmployeeById(id);
		}
		
	}
		
	\end{lstlisting}	
	\section{Testing with Rest Client [ARC]}
	\newpage
	\section{Spring Boot Framework Plug-in Eclipse}
	\section{Assignments for practice}
\end{document}

